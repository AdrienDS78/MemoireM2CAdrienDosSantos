\chapter{Bilan}


\section{Conclusion}

La compréhension de l’automatisation du langage théorique existe par le NLP découpé en deux parties :  le NLU qui utilise le TAL pour toute la partie analyse des phrases et du dialogue ainsi que le NLG pour la partie raisonnement après analyse. 
\vspace{1em}

Cette compréhension a été utile dans la phase de conception du scénario pour obtenir un dialogue simple et efficace. Pour la partie paramétrage, les intents simplifient  le langage naturel avec un outil qui permet une meilleure correspondance avec ce qui est possible de faire et les utterances améliorent cette performance avec la phase de machine learning. Le NLU est géré mis à part le discours et le sens des phrases qui eux sont à gérer avec des algorithmes car il n’existe pas d’outils pour les traiter. Quant au NLG, une base de connaissances, le CRM et le machine Learning avec la plateforme ont suffi.
\vspace{1em}

Cette solution est encore en constante évolution notamment avec les différents langages, l’anglais est très bien maîtrisé alors que le reste beaucoup moins voire pas du tout dans les analyses syntaxiques par exemple.
\vspace{1em}

Au vu de cette solution, la compréhension a permis la bonne conceptualisation, modélisation et le bon paramétrage d’un bot de type chatbot. Cette compréhension a été le plus utile pour nous en amont du paramétrage et sur le NLP.
\vspace{1em}

Maintenant, si on veut parler d’une IA qui sera un domaine ouvert,  la compréhension pour le paramétrage restera le même que sur cette solution pour le NLP tout du moins. Le NLG sera la partie principale d’une IA avec une réflexion primordiale où la compréhension du langage trouvera très vite une limite.

\section{Evolution possibles}

Durant ce mémoire, nous nous sommes focalisés sur un dialogue écrit mais bien évidemment celui oral est tout aussi important et utilise les mêmes principes à la différence qu’on a une partie de retranscription de ce qui est dit pour ensuite analyse. Cela aurait un impact supplémentaire sur le paramétrage.
\vspace{1em}

On pourrait avoir une plus grande diversité des langues avec par exemple une phase où on analyserait le langage de l’utilisateur dès le début d’un dialogue même si cela reste compliqué vu les problèmes actuels pour passer de l’une à l’autre. La partie NLU serait encore plus complexe avec une compréhension du langage pour le paramétrage.
\vspace{1em}

Il y a aussi un dialogue avec un bot réel où des interactions physiques pourraient avoir lieu avec une discussion vocale pour le NLU et NLG,  par exemple pour déplacer un objet d’un point A à un point B de la meilleure des façons. 
\vspace{1em}

Une partie non évoquée, c’est le cognitif avec une détection des expressions d’une personne en plus de ce qu’il pourrait dire. On aurait une partie raisonnement plus poussée pour comprendre les sentiments de la personne et répondre plus intelligemment.
\vspace{1em}

Pour rendre un bot encore plus intelligent, on pourrait utiliser un réseau neuronal artificiel qui est un ensemble d’algorithme pour formuler des hypothèses et faire un raisonnement avec des informations issues du deep learning pour un apprentissage automatique.
\vspace{1em}

Toutes les évolutions possibles présentées pourraient être dans un même bot et on se rapprocherait d'un bot plus proche de l'homme. 
