\chapter*{Introduction}
\markboth{Introduction}{}

Depuis les années $50$, avec le test d’Alan Turing pour tester le niveau d’intelligence artificielle d’un robot, on ne cesse d’inventer et de créer des robots qui sont de plus en plus performants dans la manière d’analyser et de répondre dans un dialogue homme-machine. Le but recherché dans l’évolution de ce dialogue est d’arriver à un point où on aurait du mal à distinguer si on dialogue avec un robot ou une personne \cite{ref1}.
\vspace{1em}

	En parallèle de l’intelligence artificielle, la technologie continue à évoluer. On a maintenant une expansion des smartphones et d’accès à internet, ce qui ouvre à un très grand public. Sur ces supports, on retrouve de l’IA que ce soit des agents virtuels sur internet comme Amélia de chez IPsoft, des assistants vocaux comme Siri de Apple, des moteurs de recherche comme Google et divers chatbots sur différents canaux notamment Facebook messenger.
	A travers tous ces agents de conversations, le but recherché est de répondre le plus précisément possible aux questions des utilisateurs.
\vspace{1em}

	Ayant été amené à travailler sur les chatbots lors de mon stage de fin d’études, j’ai souhaité approfondir le sujet dans le cadre du mémoire. 
	\vspace{1em}

	On se dirige vers un avenir où l’IA sera omniprésente car elle sera plus ou moins aussi intelligente qu’un humain dans sa manière de raisonner et pourra exécuter des tâches humaines compliquées ou physiquement dures.
	\vspace{1em}
	
	On peut observer la victoire de l’IA AlphaGO de la société DeepMind, filiale de Google, battre le champion du monde du jeu de go en Mars $2017$ où « le nombre de combinaisons dans ce jeu est supérieur au nombre d’atomes que compte l’univers » d’après Demis Hassabis, l’un des fondateurs de la société.
	\vspace{1em}
	
	Ce sujet m’intéresse car il consiste à comprendre comment arriver à créer un bot qui répond à un besoin en fonction du dialogue que l’on va utiliser.
	\vspace{1em}

	Avec les avancées scientifiques, on a fait évoluer ce dialogue en le rendant plus complexe avec plus de choix en entrées et plus de réflexion pour répondre aux besoins des utilisateurs. Le problème réside dans la compréhension de ce que dit ou fait l’utilisateur. Tout au long de ce mémoire, nous étudierons les moyens possibles de compréhension du dialogue et les messages générés ;  Le but étant de partir de l’existant pour s’en servir comme support et de comprendre jusqu’où la compréhension du dialogue a permis de créer un bot. L’analyse des différentes études nous permettra de mettre en évidence les améliorations et les limites du paramétrage d’un bot.


\clearpage

